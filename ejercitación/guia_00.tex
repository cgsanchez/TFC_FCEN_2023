\documentclass[a4paper,12pt]{article}
\usepackage[spanish]{babel}
\usepackage[utf8]{inputenc}
\usepackage{graphicx}
\usepackage{booktabs}
\usepackage{mathtools}
\mathtoolsset{showonlyrefs} 
\usepackage{amsmath}
\usepackage{amsfonts}

\usepackage{hyperref}

\hypersetup{
    colorlinks = true,
    allcolors = blue
}

\begin{document}
\title{Taller de Física Computacional}

\author{Guía 0: Precalentamiento ({\em allegro cantabile})}

\date{\today}

\maketitle

\section*{Ejercicios}
\vspace{0.5cm}
\begin{enumerate}
    \item[\bf Ejercicio 1] Dadas las siguientes expresiones para la función exponencial 
    \begin{equation}
    	e^x = \sum_0^\infty \frac{x^k}{k!}
    \end{equation}
    \vspace{2mm}
    \begin{equation}
    	e^x = \lim_{n \to \infty} \left( 1 + \frac{x}{n}\right)^n
    \end{equation} 
    \vspace{2mm}
    \begin{equation}
    e^{x}=1+\frac{x}{1-\frac{x}{x+2-\frac{2 x}{x+3-\frac{3 x}{x+4-\ddots}}}}	
    \end{equation}

    	Implementar una función que calcule el número $e$ con al menos 13 cifras correctas usando cada una de ellas.
    	
    \item[\bf Ejercicio 2]
    	
    	Implementar una función que dado un número entero positivo de por lo menos 128 bits devuelva una cadena de caracteres con su representación en base 3. Implementar también la inversa.
    	
    \item[\bf Ejercicio 3] 
    
    (*) Si aún no sitió fluir su líquido cerebroespinal extienda el código del ejercicio 3 para enteros positivos sin cota superior. 
    
    \item[\bf Ejercicio 4] 
    
    (**) Implementar una función que encuentre los primeros $n$ pares de primos gemelos. 
    
\end{enumerate}

\begin{center}
    $\clubsuit$~$\clubsuit$~$\clubsuit$
  \end{center}




\end{document}
